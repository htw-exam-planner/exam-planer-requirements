\PassOptionsToPackage{unicode=true}{hyperref} % options for packages loaded elsewhere
\PassOptionsToPackage{hyphens}{url}
%
\documentclass[]{article}
\usepackage{lmodern}
\usepackage{amssymb,amsmath}
\usepackage{ifxetex,ifluatex}
\usepackage{fixltx2e} % provides \textsubscript
\ifnum 0\ifxetex 1\fi\ifluatex 1\fi=0 % if pdftex
  \usepackage[T1]{fontenc}
  \usepackage[utf8]{inputenc}
  \usepackage{textcomp} % provides euro and other symbols
\else % if luatex or xelatex
  \usepackage{unicode-math}
  \defaultfontfeatures{Ligatures=TeX,Scale=MatchLowercase}
\fi
% use upquote if available, for straight quotes in verbatim environments
\IfFileExists{upquote.sty}{\usepackage{upquote}}{}
% use microtype if available
\IfFileExists{microtype.sty}{%
\usepackage[]{microtype}
\UseMicrotypeSet[protrusion]{basicmath} % disable protrusion for tt fonts
}{}
\IfFileExists{parskip.sty}{%
\usepackage{parskip}
}{% else
\setlength{\parindent}{0pt}
\setlength{\parskip}{6pt plus 2pt minus 1pt}
}
\usepackage{hyperref}
\hypersetup{
            pdfborder={0 0 0},
            breaklinks=true}
\urlstyle{same}  % don't use monospace font for urls
\usepackage{longtable,booktabs}
% Fix footnotes in tables (requires footnote package)
\IfFileExists{footnote.sty}{\usepackage{footnote}\makesavenoteenv{longtable}}{}
\usepackage{graphicx,grffile}
\makeatletter
\def\maxwidth{\ifdim\Gin@nat@width>\linewidth\linewidth\else\Gin@nat@width\fi}
\def\maxheight{\ifdim\Gin@nat@height>\textheight\textheight\else\Gin@nat@height\fi}
\makeatother
% Scale images if necessary, so that they will not overflow the page
% margins by default, and it is still possible to overwrite the defaults
% using explicit options in \includegraphics[width, height, ...]{}
\setkeys{Gin}{width=\maxwidth,height=\maxheight,keepaspectratio}
\setlength{\emergencystretch}{3em}  % prevent overfull lines
\providecommand{\tightlist}{%
  \setlength{\itemsep}{0pt}\setlength{\parskip}{0pt}}
\setcounter{secnumdepth}{0}
% Redefines (sub)paragraphs to behave more like sections
\ifx\paragraph\undefined\else
\let\oldparagraph\paragraph
\renewcommand{\paragraph}[1]{\oldparagraph{#1}\mbox{}}
\fi
\ifx\subparagraph\undefined\else
\let\oldsubparagraph\subparagraph
\renewcommand{\subparagraph}[1]{\oldsubparagraph{#1}\mbox{}}
\fi

% set default figure placement to htbp
\makeatletter
\def\fps@figure{htbp}
\makeatother


\date{}

\begin{document}

\hypertarget{pflichtenheft}{%
\section{Pflichtenheft}\label{pflichtenheft}}

Belegarbeit Software-Engineering II

Sommersemester 2018

Thema: Entwicklung eines SW‐Systems zur Unterstützung der Planung von
Präsentationen im Kontext des Moduls SE II

Bearbeitet von: * Denis Keiling * Fabian Krehnke * Leo Lindhorst * Đức
Hùng Nguyễn * Oliver von Seydlitz

Auftraggeber: Prof.~Dr.~Hauptmann

\hypertarget{problemstellung-und-systemziel}{%
\subsection{Problemstellung und
Systemziel}\label{problemstellung-und-systemziel}}

\hypertarget{problem}{%
\subsubsection{Problem}\label{problem}}

Die Terminbelegung für Prüfungspräsentationen über E-Mail und manuelles
Editieren der Excel-Tabelle ist zeitaufwändig und kann zu Fehlern
führen.

\hypertarget{ziel}{%
\subsubsection{Ziel}\label{ziel}}

Es soll ein Software-System entwickelt werden, über das die
Terminplanung geregelt wird. Es soll folgendes gewährleisten: * Ein
gleichzeitiges Buchen darf nicht möglich sein um Konflikte zu vermeiden
* Die Termine müssen vom Prüfer flexibel angegeben werden können *
Organisation muss dezentral sein um den Prüfer zu entlasten

\hypertarget{systemkontext}{%
\subsection{Systemkontext}\label{systemkontext}}

Der Kontext des Systems wird durch folgendes Diagramm beschrieben:

\begin{figure}
\centering
\includegraphics{diagrams/awf-kontext.jpg}
\caption{Kontexdiagramm}
\end{figure}

\hypertarget{benutzerschnittstelle}{%
\subsection{Benutzerschnittstelle}\label{benutzerschnittstelle}}

\hypertarget{startbildschirm---rollenauswahl}{%
\subsubsection{Startbildschirm -
Rollenauswahl}\label{startbildschirm---rollenauswahl}}

Auf dem Startbildschirm kann der Benutzer auswählen, ob die Anwendung im
Administrator- oder im Studentenmodus genutzt werden soll.
\includegraphics{ui-prototypes/exports/Role selection.png}

\hypertarget{studentenansicht}{%
\subsubsection{Studentenansicht}\label{studentenansicht}}

\hypertarget{gruppenauswahl}{%
\paragraph{Gruppenauswahl}\label{gruppenauswahl}}

In der Gruppenauswahl kann der Student die Gruppe auswählen, für die er
die Buchung durchführen möchte.
\includegraphics{ui-prototypes/exports/Group selection.png}

\hypertarget{terminansicht}{%
\paragraph{Terminansicht}\label{terminansicht}}

In der Terminansicht kann sich der Student eine Übersicht über die
verfügbaren Termine veschaffen und Buchungen/Reservierungen verwalten.
\includegraphics{ui-prototypes/exports/Date management.png}

\hypertarget{buchung-durchfuxfchren}{%
\subparagraph{Buchung durchführen}\label{buchung-durchfuxfchren}}

Nach dem Drücken des ``Buchen'' Knopfes muss der Student die Startzeit
für die Prüfung eingeben.
\includegraphics{ui-prototypes/exports/Date management (Book Button pressed).png}

\hypertarget{administratoransicht}{%
\subsubsection{Administratoransicht}\label{administratoransicht}}

\hypertarget{gruppen-und-termine-anlegen}{%
\paragraph{Gruppen und Termine
anlegen}\label{gruppen-und-termine-anlegen}}

Mithilfe des Formulars kann der Administrator die initialen Gruppen und
Termine anlegen. \includegraphics{ui-prototypes/exports/Setup.png}

\hypertarget{terminansicht-1}{%
\paragraph{Terminansicht}\label{terminansicht-1}}

In der Terminansicht kann sich der Administrator eine Übersicht über die
verfügbaren Termine veschaffen, Termine editieren und
Buchungen/Reservierungen verwalten.
\includegraphics{ui-prototypes/exports/Date management Admin.png}

\hypertarget{termin-bearbeiten}{%
\subparagraph{Termin bearbeiten}\label{termin-bearbeiten}}

Durch die Schaltfläche ``Bearbeiten'', kann der Administrator den
Zeitraum und den Kommentar des Termins bearbeiten und den Termin
deaktivieren.
\includegraphics{ui-prototypes/exports/Date management Admin (Edit Button pressed, no Booking).png}

Falls eine Buchung für den Termin besteht, kann der Administrator die
Buchung stornieren oder den Zeitraum sowie Prüfungsraum der Buchung
festlegen.
\includegraphics{ui-prototypes/exports/Date management Admin (Edit Button pressed, Booking present).png}

Falls eine Reservierung für den Termin besteht, kann der Administrator
die Reservierung stornieren.
\includegraphics{ui-prototypes/exports/Date management Admin (Edit Button pressed, Reservation present).png}

Falls der Termin deaktiviert ist, kann der Administrator den Termin
wieder aktivieren.
\includegraphics{ui-prototypes/exports/Date management Admin (Edit Button pressed, Deactivated).png}

\hypertarget{gruppenansicht}{%
\paragraph{Gruppenansicht}\label{gruppenansicht}}

In der Gruppenansicht kann sich der Administrator einen Überblick über
den Zustand der Gruppen verschaffen, sowie Gruppen mithilfe der
entsprechenden Schaltflächen anlegen und löschen.
\includegraphics{ui-prototypes/exports/Group management.png}

\hypertarget{anforderungen}{%
\subsection{Anforderungen}\label{anforderungen}}

\hypertarget{funktionale-anforderungen}{%
\subsubsection{Funktionale
Anforderungen}\label{funktionale-anforderungen}}

\hypertarget{uxfcberblick}{%
\paragraph{Überblick}\label{uxfcberblick}}

Die folgende Tabelle gibt einen Überblick über die essentiellen Gruppen
und essentiellen Funktionen, sowie deren Auslöser und Reaktionen:

\begin{longtable}[]{@{}lll@{}}
\toprule
\begin{minipage}[b]{0.40\columnwidth}\raggedright
Anwendungsfall\strut
\end{minipage} & \begin{minipage}[b]{0.27\columnwidth}\raggedright
Auslöser/Eingabe\strut
\end{minipage} & \begin{minipage}[b]{0.24\columnwidth}\raggedright
Reaktion/Ausgabe\strut
\end{minipage}\tabularnewline
\midrule
\endhead
\begin{minipage}[t]{0.40\columnwidth}\raggedright
\textbf{Prüfungstermine verwalten}\strut
\end{minipage} & \begin{minipage}[t]{0.27\columnwidth}\raggedright
\strut
\end{minipage} & \begin{minipage}[t]{0.24\columnwidth}\raggedright
\strut
\end{minipage}\tabularnewline
\begin{minipage}[t]{0.40\columnwidth}\raggedright
Termine und Gruppen generieren\strut
\end{minipage} & \begin{minipage}[t]{0.27\columnwidth}\raggedright
Startdatum + Anzahl Gruppen\strut
\end{minipage} & \begin{minipage}[t]{0.24\columnwidth}\raggedright
Termine + Gruppen\strut
\end{minipage}\tabularnewline
\begin{minipage}[t]{0.40\columnwidth}\raggedright
Termin bearbeiten\strut
\end{minipage} & \begin{minipage}[t]{0.27\columnwidth}\raggedright
Termin\strut
\end{minipage} & \begin{minipage}[t]{0.24\columnwidth}\raggedright
Termine bzw. Warnhinweis\strut
\end{minipage}\tabularnewline
\begin{minipage}[t]{0.40\columnwidth}\raggedright
Termin als frei markieren\strut
\end{minipage} & \begin{minipage}[t]{0.27\columnwidth}\raggedright
Termin\strut
\end{minipage} & \begin{minipage}[t]{0.24\columnwidth}\raggedright
Termine\strut
\end{minipage}\tabularnewline
\begin{minipage}[t]{0.40\columnwidth}\raggedright
Termin deaktivieren\strut
\end{minipage} & \begin{minipage}[t]{0.27\columnwidth}\raggedright
Termin\strut
\end{minipage} & \begin{minipage}[t]{0.24\columnwidth}\raggedright
Termine\strut
\end{minipage}\tabularnewline
\begin{minipage}[t]{0.40\columnwidth}\raggedright
Zeitfenster bearbeiten\strut
\end{minipage} & \begin{minipage}[t]{0.27\columnwidth}\raggedright
Termin\strut
\end{minipage} & \begin{minipage}[t]{0.24\columnwidth}\raggedright
Termine bzw. Warnhinweis\strut
\end{minipage}\tabularnewline
\begin{minipage}[t]{0.40\columnwidth}\raggedright
Bemerkung bearbeiten\strut
\end{minipage} & \begin{minipage}[t]{0.27\columnwidth}\raggedright
Termin\strut
\end{minipage} & \begin{minipage}[t]{0.24\columnwidth}\raggedright
Termine\strut
\end{minipage}\tabularnewline
\begin{minipage}[t]{0.40\columnwidth}\raggedright
Raum und Endzeit festlegen\strut
\end{minipage} & \begin{minipage}[t]{0.27\columnwidth}\raggedright
Termin\strut
\end{minipage} & \begin{minipage}[t]{0.24\columnwidth}\raggedright
Termine bzw. Warnhinweis\strut
\end{minipage}\tabularnewline
\begin{minipage}[t]{0.40\columnwidth}\raggedright
Termine exportieren\strut
\end{minipage} & \begin{minipage}[t]{0.27\columnwidth}\raggedright
\strut
\end{minipage} & \begin{minipage}[t]{0.24\columnwidth}\raggedright
Termine\strut
\end{minipage}\tabularnewline
\begin{minipage}[t]{0.40\columnwidth}\raggedright
\textbf{Termine anzeigen}\strut
\end{minipage} & \begin{minipage}[t]{0.27\columnwidth}\raggedright
\strut
\end{minipage} & \begin{minipage}[t]{0.24\columnwidth}\raggedright
\strut
\end{minipage}\tabularnewline
\begin{minipage}[t]{0.40\columnwidth}\raggedright
Termine anzeigen\strut
\end{minipage} & \begin{minipage}[t]{0.27\columnwidth}\raggedright
\strut
\end{minipage} & \begin{minipage}[t]{0.24\columnwidth}\raggedright
Termine\strut
\end{minipage}\tabularnewline
\begin{minipage}[t]{0.40\columnwidth}\raggedright
\textbf{Gruppen verwalten}\strut
\end{minipage} & \begin{minipage}[t]{0.27\columnwidth}\raggedright
\strut
\end{minipage} & \begin{minipage}[t]{0.24\columnwidth}\raggedright
\strut
\end{minipage}\tabularnewline
\begin{minipage}[t]{0.40\columnwidth}\raggedright
Gruppe anlegen\strut
\end{minipage} & \begin{minipage}[t]{0.27\columnwidth}\raggedright
Gruppenname\strut
\end{minipage} & \begin{minipage}[t]{0.24\columnwidth}\raggedright
Gruppe\strut
\end{minipage}\tabularnewline
\begin{minipage}[t]{0.40\columnwidth}\raggedright
Gruppe löschen\strut
\end{minipage} & \begin{minipage}[t]{0.27\columnwidth}\raggedright
Gruppe\strut
\end{minipage} & \begin{minipage}[t]{0.24\columnwidth}\raggedright
Gruppen (neu)\strut
\end{minipage}\tabularnewline
\begin{minipage}[t]{0.40\columnwidth}\raggedright
Gruppen anzeigen\strut
\end{minipage} & \begin{minipage}[t]{0.27\columnwidth}\raggedright
\strut
\end{minipage} & \begin{minipage}[t]{0.24\columnwidth}\raggedright
Gruppen\strut
\end{minipage}\tabularnewline
\begin{minipage}[t]{0.40\columnwidth}\raggedright
\textbf{Prüfungstermine anzeigen und buchen}\strut
\end{minipage} & \begin{minipage}[t]{0.27\columnwidth}\raggedright
\strut
\end{minipage} & \begin{minipage}[t]{0.24\columnwidth}\raggedright
\strut
\end{minipage}\tabularnewline
\begin{minipage}[t]{0.40\columnwidth}\raggedright
Als Gruppe anmelden\strut
\end{minipage} & \begin{minipage}[t]{0.27\columnwidth}\raggedright
Gruppe\strut
\end{minipage} & \begin{minipage}[t]{0.24\columnwidth}\raggedright
Termine\strut
\end{minipage}\tabularnewline
\begin{minipage}[t]{0.40\columnwidth}\raggedright
Termin reservieren\strut
\end{minipage} & \begin{minipage}[t]{0.27\columnwidth}\raggedright
Gruppe + Termin\strut
\end{minipage} & \begin{minipage}[t]{0.24\columnwidth}\raggedright
ggf. Warnhinweis\strut
\end{minipage}\tabularnewline
\begin{minipage}[t]{0.40\columnwidth}\raggedright
Reservierten Termin stornieren\strut
\end{minipage} & \begin{minipage}[t]{0.27\columnwidth}\raggedright
Termin\strut
\end{minipage} & \begin{minipage}[t]{0.24\columnwidth}\raggedright
ggf. Warnhinweis\strut
\end{minipage}\tabularnewline
\begin{minipage}[t]{0.40\columnwidth}\raggedright
Termin buchen\strut
\end{minipage} & \begin{minipage}[t]{0.27\columnwidth}\raggedright
Gruppe + Termin + Startzeit\strut
\end{minipage} & \begin{minipage}[t]{0.24\columnwidth}\raggedright
ggf. Warnhinweis\strut
\end{minipage}\tabularnewline
\bottomrule
\end{longtable}

Die essentiellen Gruppen sind ferner in folgendem Anwendungsfalldiagramm
dargestellt. Abstrakte Anwendungsfälle sind dabei kursiv geschrieben.

\begin{figure}
\centering
\includegraphics{diagrams/awf-ueberblick.jpg}
\caption{AWF-Diagramm Überblick}
\end{figure}

\hypertarget{struktur-der-ausluxf6ser-und-reaktionen}{%
\subparagraph{Struktur der Auslöser und
Reaktionen}\label{struktur-der-ausluxf6ser-und-reaktionen}}

Termin = Datum + Zeitfenster + Terminzustand + Bemerkung + (
Reservierung \textbar{} Buchung )

Termine = \{Termin\}

Zeitfenster = Startzeit + Endzeit

Terminzustand = frei \textbar{} deaktiviert \textbar{} reserviert
\textbar{} gebucht

Reservierung = Gruppe

Buchung = Gruppe + Startzeit + (Endzeit) + (Raum)

Gruppe = Gruppenname + Buchungsstatus

Buchungsstatus = gebucht \textbar{} noch nicht gebucht

Gruppen = \{Gruppe\}

Die anderen Auslöser und Reaktionen sind atomare Werte.

Es folgen nun die Beschreibungen aller Anwendungsfälle.

\hypertarget{termine-verwalten}{%
\paragraph{Termine verwalten}\label{termine-verwalten}}

Folgendes Anwendungsfalldiagramm gibt einen Überblick über die
Anwendungsfälle der essentiellen Gruppe ``Termine verwalten'':

\begin{figure}
\centering
\includegraphics{diagrams/awf-termine-verwalten.jpg}
\caption{AWF-Diagramm Termine verwalten Lehrende}
\end{figure}

\hypertarget{termine-und-gruppen-generieren}{%
\subparagraph{Termine und Gruppen
generieren}\label{termine-und-gruppen-generieren}}

Die Funktion ``Termine und Gruppen generieren'' * muss dem Administrator
die Möglichkeit bieten, die 15 Termine eines Prüfungszeitraums sowie die
Gruppen dieses Zeitraums anzulegen.

Folgendes Aktivitätsdiagramm verdeutlicht dies:

\begin{figure}
\centering
\includegraphics{diagrams/ad-termine-gruppen-generieren.jpg}
\caption{Aktivitätsdiagramm Termine und Gruppen generieren}
\end{figure}

\hypertarget{termin-bearbeiten-1}{%
\subparagraph{Termin bearbeiten}\label{termin-bearbeiten-1}}

Die Funktion ``Termin bearbeiten'' * muss dem Administrator die
Möglichkeit bieten, eine der folgenden Aktionen durchzuführen: * Termin
als frei markieren * Termin deaktivieren * Zeitfenster bearbeiten *
Bemerkung bearbeiten * Raum und Endzeit festlegen

Folgendes Aktivitätsdiagramm verdeutlicht dies. Die möglichen Aktionen
sind in ihren jeweiligen Aktivitätsdiagrammen beschrieben:

\begin{figure}
\centering
\includegraphics{diagrams/ad-termin-bearbeiten.jpg}
\caption{Aktivitätsdiagramm Termin frei markieren}
\end{figure}

\hypertarget{termin-als-frei-markieren}{%
\subparagraph{Termin als frei
markieren}\label{termin-als-frei-markieren}}

Die Funktion ``Termin als frei markieren'' * muss dem Administrator die
Möglichkeit bieten, einen Termin als frei zu markieren. * muss beim frei
markieren eine ggf. vorhandene Reservierung oder Buchung löschen.

Folgendes Aktivitätsdiagramm verdeutlicht dies:

\begin{figure}
\centering
\includegraphics{diagrams/ad-termin-frei-markieren.jpg}
\caption{Aktivitätsdiagramm Termin frei markieren}
\end{figure}

\hypertarget{termin-deaktivieren}{%
\subparagraph{Termin deaktivieren}\label{termin-deaktivieren}}

Die Funktion ``Termin deaktivieren'' * muss dem Administrator die
Möglichkeit bieten, einen Termin zu deaktivieren. * muss beim
Deaktivieren eine ggf. vorhandene Reservierung oder Buchung löschen

Folgendes Aktivitätsdiagramm verdeutlicht dies:

\begin{figure}
\centering
\includegraphics{diagrams/ad-termin-deaktivieren.jpg}
\caption{Aktivitätsdiagramm Termin deaktivieren}
\end{figure}

\hypertarget{zeitfenster-bearbeiten}{%
\subparagraph{Zeitfenster bearbeiten}\label{zeitfenster-bearbeiten}}

Die Funktion ``Zeitfenster bearbeiten'' * muss dem Administrator die
Möglichkeit bieten, das Zeitfenster eines Termins zu bearbeiten. * muss
prüfen, ob beim angegebenen Zeitfenster die End- nach der Startzeit
liegt.

Folgendes Aktivitätsdiagramm verdeutlicht dies:

\begin{figure}
\centering
\includegraphics{diagrams/ad-zeitfenster-bearbeiten.jpg}
\caption{Aktivitätsdiagramm Zeitfenster bearbeiten}
\end{figure}

\hypertarget{bemerkung-bearbeiten}{%
\subparagraph{Bemerkung bearbeiten}\label{bemerkung-bearbeiten}}

Die Funktion ``Bemerkung bearbeiten'' * muss dem Administrator die
Möglichkeit bieten, einem Termin eine Bemerkung hinzuzufügen. * muss dem
Administrator die Möglichkeit bieten, die Bemerkung eines Termins,
sofern vorhanden, zu bearbeiten.

Folgendes Aktivitätsdiagramm verdeutlicht dies:

\begin{figure}
\centering
\includegraphics{diagrams/ad-bemerkung-bearbeiten.jpg}
\caption{Aktivitätsdiagramm Zeitfenster bearbeiten}
\end{figure}

\hypertarget{raum-und-endzeit-einer-buchung-festlegen}{%
\subparagraph{Raum und Endzeit einer Buchung
festlegen}\label{raum-und-endzeit-einer-buchung-festlegen}}

Die Funktion ``Raum und Endzeit einer Buchung festlegen'' * muss dem
Administrator die Möglichkeit bieten, einem gebuchten Prüfungstermin den
Prüfungsraum und die Endzeit der Prüfung zuzuordnen. * muss
sicherstellen, dass die Endzeit nach der Startzeit liegt.

Folgendes Aktivitätsdiagramm verdeutlicht dies:

\begin{figure}
\centering
\includegraphics{diagrams/ad-raum-endzeit-festlegen.jpg}
\caption{Aktivitätsdiagramm Raum und Endzeit festlegen}
\end{figure}

\hypertarget{termine-exportieren}{%
\subparagraph{Termine exportieren}\label{termine-exportieren}}

Die Funktion ``Termine exportieren'' * sollte dem Administrator die
Möglichkeit bieten, eine Übersicht der gebuchten Termine als PDF zu
exportieren. * sollte dem Administrator die Möglichkeit bieten, eine
Übersicht der gebuchten Termine als Kalenderdaten zu exportieren.

Folgendes Aktivitätsdiagramm verdeutlicht dies:

\begin{figure}
\centering
\includegraphics{diagrams/ad-termine-exportieren.jpg}
\caption{Aktivitätsdiagramm Termine exportieren}
\end{figure}

\hypertarget{gruppen-verwalten}{%
\paragraph{Gruppen verwalten}\label{gruppen-verwalten}}

Folgendes Anwendungsfalldiagramm gibt einen Überblick über die
Anwendungsfälle der essentiellen Gruppe ``Gruppen verwalten'':

\begin{figure}
\centering
\includegraphics{diagrams/awf-gruppen-verwalten.jpg}
\caption{AWF-Diagramm Gruppen verwalten}
\end{figure}

\hypertarget{gruppe-anlegen}{%
\subparagraph{Gruppe anlegen}\label{gruppe-anlegen}}

Die Funktion ``Gruppe anlegen'' * muss dem Administrator die Möglichkeit
bieten, eine Gruppe anzulegen.

Folgendes Aktivitätsdiagramm verdeutlicht dies:

\begin{figure}
\centering
\includegraphics{diagrams/ad-gruppe-anlegen.jpg}
\caption{Aktivitätsdiagramm Gruppe anlegen}
\end{figure}

\hypertarget{gruppe-luxf6schen}{%
\subparagraph{Gruppe löschen}\label{gruppe-luxf6schen}}

Die Funktion ``Gruppe löschen'' * muss dem Administrator die Möglichkeit
bieten, eine Gruppe zu löschen.

Folgendes Aktivitätsdiagramm verdeutlicht dies:

\begin{figure}
\centering
\includegraphics{diagrams/ad-gruppe-loeschen.jpg}
\caption{Aktivitätsdiagramm Gruppe löschen}
\end{figure}

\hypertarget{gruppen-anzeigen}{%
\subparagraph{Gruppen anzeigen}\label{gruppen-anzeigen}}

Die Funktion ``Gruppen anzeigen'' * muss dem Administrator die
Möglichkeit bieten, alle Gruppen anzuzeigen. * muss die Gruppen, die
noch nicht gebucht haben, visuell hervorheben.

Folgendes Aktivitätsdiagramm verdeutlicht dies:

\begin{figure}
\centering
\includegraphics{diagrams/ad-gruppen-anzeigen.jpg}
\caption{Aktivitätsdiagramm Gruppen anzeigen}
\end{figure}

\hypertarget{termine-anzeigen}{%
\paragraph{Termine anzeigen}\label{termine-anzeigen}}

Die Funktion ``Termine anzeigen'' * muss dem Administrator und den
Studenten die Möglichkeit bieten, die Termine anzuzeigen.

\begin{figure}
\centering
\includegraphics{diagrams/ad-termine-anzeigen.jpg}
\caption{AWF-Diagramm Termine anzeigen}
\end{figure}

\hypertarget{termine-buchen-und-reservieren}{%
\paragraph{Termine buchen und
reservieren}\label{termine-buchen-und-reservieren}}

Folgendes Anwendungsfalldiagramm gibt einen Überblick über die
Anwendungsfälle der essentiellen Gruppe ``Termine buchen und
reservieren'':

\begin{figure}
\centering
\includegraphics{diagrams/awf-termine-reservieren-und-buchen.jpg}
\caption{AWF-Diagramm Termine buchen und reservieren}
\end{figure}

\hypertarget{als-gruppe-anmelden}{%
\subparagraph{Als Gruppe anmelden}\label{als-gruppe-anmelden}}

Die Funktion ``Als Gruppe anmelden'' * muss einem Studenten ermöglichen,
sich unter Angabe seiner Gruppe am System anzumelden.

\begin{figure}
\centering
\includegraphics{diagrams/ad-gruppe-anmelden.jpg}
\caption{Aktivitätsdiagramm Als Gruppe anmelden}
\end{figure}

\hypertarget{termin-reservieren}{%
\subparagraph{Termin reservieren}\label{termin-reservieren}}

Die Funktion ``Termin reservieren'' * muss einem Studenten die
Möglichkeit bieten, einen Termin zu reservieren. * muss sicherstellen,
dass eine Gruppe nur eine Reservierung tätigen kann. * muss
sicherstellen, dass nur freie Termine reserviert werden können. * muss
sicherstellen, dass eine Gruppe, die bereits gebucht hat, keinen Termin
mehr reservieren kann.

Folgendes Aktivitätsdiagramm verdeutlicht dies:

\begin{figure}
\centering
\includegraphics{diagrams/ad-termin-reservieren.jpg}
\caption{Aktivitätsdiagramm Termin reservieren}
\end{figure}

\hypertarget{reservierung-stornieren}{%
\subparagraph{Reservierung stornieren}\label{reservierung-stornieren}}

Die Funktion ``Reservierung stornieren'' * muss einem Studenten die
Möglichkeit bieten, die Reservierung eines Termins zu stornieren.

Folgendes Aktivitätsdiagramm verdeutlicht dies:

\begin{figure}
\centering
\includegraphics{diagrams/ad-reservierung-stornieren.jpg}
\caption{Aktivitätsdiagramm Reservierung stornieren}
\end{figure}

\hypertarget{termin-buchen}{%
\subparagraph{Termin buchen}\label{termin-buchen}}

Die Funktion ``Termin buchen'' * muss einem Studenten die Möglichkeit
bieten, einen Prüfungstermin zu buchen. * muss sicherstellen, dass eine
Gruppe nur einen Termin buchen kann. * muss beim Buchen eine
Reservierung der buchenden Gruppe, sofern vorhanden, entfernen. * muss
sicherstellen, dass nur freie Termine gebucht werden können. * muss
prüfen, ob die gewählte Startzeit im Zeitfenster des Termins liegt.

Folgendes Aktivitätsdiagramm verdeutlicht dies:

\begin{figure}
\centering
\includegraphics{diagrams/ad-termin-buchen.jpg}
\caption{Aktivitätsdiagramm Termin buchen}
\end{figure}

\hypertarget{zustandsdiagramm-eines-termins}{%
\paragraph{Zustandsdiagramm eines
Termins}\label{zustandsdiagramm-eines-termins}}

Folgendes Zustandsdiagramm gibt einen Überblick über die Zustände eines
Termins und die möglichen Übergänge zwischen diesen Zuständen. Eine
Erklärung der Bedeutung der Zustände findet sich im Glossar.

\begin{figure}
\centering
\includegraphics{diagrams/zd-termine.jpg}
\caption{Zustandsdiagramm Termine}
\end{figure}

\hypertarget{qualituxe4tsanforderungen}{%
\subsubsection{Qualitätsanforderungen}\label{qualituxe4tsanforderungen}}

\begin{itemize}
\tightlist
\item
  Die verschiedenen Zustände eines Termins sollen visuell voneinander
  unterscheidbar sein.
\item
  In der Gruppenübersicht sollen Gruppen, die noch nicht gebucht haben,
  visuell hervorgehoben werden.
\end{itemize}

\hypertarget{rahmenbedingungen}{%
\subsubsection{Rahmenbedingungen}\label{rahmenbedingungen}}

\hypertarget{technischtechnologische-rahmenbedingungen}{%
\paragraph{technisch/technologische
Rahmenbedingungen}\label{technischtechnologische-rahmenbedingungen}}

\begin{itemize}
\tightlist
\item
  Das System wird mit Java als Desktop-App implementiert.
\item
  Es wird ferner eine MySQL-Datenbank auf einem Hochschulserver
  verwendet. Der Auftraggeber beantragt diese bei der HTW.
\item
  Die technischen Abhängigkeiten sollen so gering wie möglich sein.
\end{itemize}

\hypertarget{organisatorische-rahmenbedingungen}{%
\paragraph{organisatorische
Rahmenbedingungen}\label{organisatorische-rahmenbedingungen}}

\begin{itemize}
\tightlist
\item
  Es wird davon ausgegangen, dass die Gruppen einander vertrauen und
  sich keine Gruppe als eine andere ausgibt.
\item
  Es wird davon ausgegangen, dass sich eine Gruppe intern abspricht,
  bevor sie einen Termin reserviert oder bucht.
\item
  Es findet höchstens eine Prüfung pro Tag statt.
\item
  Die gebuchte Startzeit ist verbindlich. Die Gruppe muss beim Buchen
  selbst dafür sorgen, dass die Startzeit früh genug innerhalb des
  Zeitrahmens gewählt wird, sodass die komplette Prüfung im Zeitrahmen
  bleibt.
\item
  Die Endzeit einer Prüfung wird von der Prüfenden eingetragen.
\item
  Das System ist nicht für die Authentifizierung der Studierenden und
  Lehrenden zuständig.
\end{itemize}

\hypertarget{rechtliche-rahmenbedingungen}{%
\paragraph{rechtliche
Rahmenbedingungen}\label{rechtliche-rahmenbedingungen}}

\begin{itemize}
\tightlist
\item
  Um dem Datenschutz gerecht zu werden, werden Gruppen nur über den
  Gruppennamen identifiziert.
\end{itemize}

\hypertarget{glossar}{%
\subsection{Glossar}\label{glossar}}

\begin{longtable}[]{@{}ll@{}}
\toprule
\begin{minipage}[b]{0.42\columnwidth}\raggedright
Begriff\strut
\end{minipage} & \begin{minipage}[b]{0.52\columnwidth}\raggedright
Bedeutung\strut
\end{minipage}\tabularnewline
\midrule
\endhead
\begin{minipage}[t]{0.42\columnwidth}\raggedright
Termin\strut
\end{minipage} & \begin{minipage}[t]{0.52\columnwidth}\raggedright
Tag, Start-und Endzeit eines Zeitfensters, in dem eine Prüfung
stattfinden kann. Innerhalb des Termins kann eine Reservierung bzw.
Buchung gemacht werden\strut
\end{minipage}\tabularnewline
\begin{minipage}[t]{0.42\columnwidth}\raggedright
deaktivierter Termin\strut
\end{minipage} & \begin{minipage}[t]{0.52\columnwidth}\raggedright
Ein Tag, an dem keine Prüfung stattfinden kann.\strut
\end{minipage}\tabularnewline
\begin{minipage}[t]{0.42\columnwidth}\raggedright
freier Termin\strut
\end{minipage} & \begin{minipage}[t]{0.52\columnwidth}\raggedright
Ein Termin, an dem eine Prüfung stattfinden kann und der noch nicht
reserviert oder gebucht ist.\strut
\end{minipage}\tabularnewline
\begin{minipage}[t]{0.42\columnwidth}\raggedright
Reservierung\strut
\end{minipage} & \begin{minipage}[t]{0.52\columnwidth}\raggedright
nicht verbindliche Kennzeichnung einer Gruppe, dass sie einen Termin
buchen möchte\strut
\end{minipage}\tabularnewline
\begin{minipage}[t]{0.42\columnwidth}\raggedright
Buchung\strut
\end{minipage} & \begin{minipage}[t]{0.52\columnwidth}\raggedright
verbindliche Entscheidung einer Gruppe, einen bestimmten Termin als
Prüfungstermin in Anspruch zu nehmen\strut
\end{minipage}\tabularnewline
\begin{minipage}[t]{0.42\columnwidth}\raggedright
Student\strut
\end{minipage} & \begin{minipage}[t]{0.52\columnwidth}\raggedright
Studierende(r) des Moduls ``Software-Engineering II'', der/die in
seiner/ihrer Gruppe in der Prüfungszeit geprüft werden muss\strut
\end{minipage}\tabularnewline
\begin{minipage}[t]{0.42\columnwidth}\raggedright
Administrator\strut
\end{minipage} & \begin{minipage}[t]{0.52\columnwidth}\raggedright
Die Lehrende bzw. Prüfende des Moduls ``Software-Engineering II'', sowie
der Praktikumsbetreuer, der ebenfalls bei einer Prüfung mitwirkt\strut
\end{minipage}\tabularnewline
\bottomrule
\end{longtable}

\end{document}
