\section{Benutzerschnittstelle}
Die Benutzerschnittstelle wird über eine Java-Anwendung realisiert. Es folgen nun die Beschreibungen sowie Prototypen einiger Ansichten.
\subsection{Startbildschirm - Rollenauswahl}
Auf dem Startbildschirm kann der Benutzer auswählen, ob die Anwendung im Administrator- oder im Studentenmodus genutzt werden soll (siehe Abbildung \ref{fig:role-selection}).
\begin{figure}
  \centering
  \includegraphics{ui-prototypes/exports/Role-selection.png}
  \caption{Oberflächenprototyp Rollenauswahl}
  \label{fig:role-selection}
\end{figure}

\subsection{Studentenansicht}

\paragraph{Gruppenauswahl}
In der Gruppenauswahl (Abbildung \ref{fig:group-selection}) kann der Student die Gruppe auswählen, für die er die Buchung durchführen möchte.
\begin{figure}
  \centering
  \includegraphics{ui-prototypes/exports/Group-selection.png}
  \caption{Oberflächenprototyp Gruppenauswahl}
  \label{fig:group-selection}
\end{figure}

\paragraph{Terminansicht}
In der Terminansicht (Abbildung \ref{fig:date-management}) kann sich der Student eine Übersicht über die
verfügbaren Termine veschaffen und Buchungen/Reservierungen verwalten.
\begin{figure}
  \centering
  \includegraphics{ui-prototypes/exports/Date-management.png}
  \caption{Oberflächenprototyp Terminansicht}
  \label{fig:date-management}
\end{figure}

\subparagraph{Buchung durchführen}
Nach dem Drücken des ``Buchen'' Knopfes muss der Student die Startzeit für die Prüfung eingeben (Abbildung \ref{fig:date-management-book-button-pressed}).
\begin{figure}
  \centering
  \includegraphics{ui-prototypes/exports/Date-management_Book-Button-pressed.png}
  \caption{Oberflächenprototyp Termin durchführen}
  \label{fig:date-management-book-button-pressed}
\end{figure}

\subsection{Administratoransicht}

\paragraph{Gruppen und Termine anlegen}
Mithilfe des Formulars (Abbildung \ref{fig:setup}) kann der Administrator die initialen Gruppen und Termine anlegen.
\begin{figure}
  \centering
  \includegraphics{ui-prototypes/exports/Setup.png}
  \caption{Oberflächenprototyp Gruppen und Termine anlegen}
  \label{fig:setup}
\end{figure}

\paragraph{Terminansicht}
In der Terminansicht (Abbildung \ref{fig:date-management-admin}) kann sich der Administrator eine Übersicht über die verfügbaren Termine veschaffen, Termine bearbeiten und Buchungen/Reservierungen verwalten.
\begin{figure}
  \centering
  \includegraphics{ui-prototypes/exports/Date-management-Admin.png}
  \caption{Oberflächenprototyp Terminansicht (Administrator)}
  \label{fig:date-management-admin}
\end{figure}

\subparagraph{Termin bearbeiten}
Durch die Schaltfläche ``Bearbeiten'' kann der Administrator den Zeitraum und den Kommentar des Termins bearbeiten und den Termin deaktivieren (Abbildung \ref{fig:edit-button-pressed-no-booking}).
\begin{figure}
  \centering
  \includegraphics{ui-prototypes/exports/Date-management-Admin_Edit-Button-pressed-no-Booking.png}
  \caption{Oberflächenprototyp Termin bearbeiten}
  \label{fig:edit-button-pressed-no-booking}
\end{figure}

Falls eine Buchung für den Termin besteht, kann der Administrator die Buchung stornieren oder den Zeitraum sowie Prüfungsraum der Buchung festlegen (Abbildung \ref{fig:edit-button-pressed-booking}).
\begin{figure}
  \centering
  \includegraphics{ui-prototypes/exports/Date-management-Admin_Edit-Button-pressed-Booking-present.png}
  \caption{Oberflächenprototyp Termin bearbeiten - Buchung}
  \label{fig:edit-button-pressed-booking}
\end{figure}

Falls eine Reservierung für den Termin besteht, kann der Administrator die Reservierung stornieren (Abbildung \ref{fig:edit-button-pressed-reservation}).
\begin{figure}
  \centering
  \includegraphics{ui-prototypes/exports/Date-management-Admin_Edit-Button-pressed-Reservation-present.png}
  \caption{Oberflächenprototyp Termin bearbeiten – Reservierung}
  \label{fig:edit-button-pressed-reservation}
\end{figure}

Falls der Termin deaktiviert ist, kann der Administrator den Termin wieder aktivieren (Abbildung \ref{fig:edit-button-pressed-deactivated}).
\begin{figure}
  \centering
  \includegraphics{ui-prototypes/exports/Date-management-Admin_Edit-Button-pressed-Deactivated.png}
  \caption{Oberflächenprototyp Termin bearbeiten – Deaktiviert}
  \label{fig:edit-button-pressed-deactivated}
\end{figure}

\paragraph{Gruppenansicht}
In der Gruppenansicht (Abbildung \ref{fig:group-management}) kann sich der Administrator einen Überblick über den Zustand der Gruppen verschaffen, sowie Gruppen mithilfe der entsprechenden Schaltflächen anlegen und löschen.
\begin{figure}
  \centering
  \includegraphics{ui-prototypes/exports/Group-management.png}
  \caption{Oberflächenprototyp Gruppenansicht}
  \label{fig:group-management}
\end{figure}
