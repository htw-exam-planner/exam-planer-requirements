\section{Benutzerschnittstelle}
\subsection{Startbildschirm - Rollenauswahl}
Auf dem Startbildschirm kann der Benutzer auswählen, ob die Anwendung im
Administrator- oder im Studentenmodus genutzt werden soll.
\begin{figure}
  \centering
  \includegraphics{ui-prototypes/exports/Role selection.png}
  \caption{Rollenauswahl}
  \label{fig:role-selection}
\end{figure}

\subsection{Studentenansicht}

\paragraph{Gruppenauswahl}
In der Gruppenauswahl kann der Student die Gruppe auswählen, für die er
die Buchung durchführen möchte.
\begin{figure}
  \centering
  \includegraphics{ui-prototypes/exports/Group selection.png}
  \caption{Gruppenauswahl}
  \label{fig:group-selection}
\end{figure}

\paragraph{Terminansicht}
In der Terminansicht kann sich der Student eine Übersicht über die
verfügbaren Termine veschaffen und Buchungen/Reservierungen verwalten.
\begin{figure}
  \centering
  \includegraphics{ui-prototypes/exports/Date management.png}
  \caption{Terminansicht}
  \label{fig:date-management}
\end{figure}

\subparagraph{Buchung durchführen}
Nach dem Drücken des ``Buchen'' Knopfes muss der Student die Startzeit
für die Prüfung eingeben.
\begin{figure}
  \centering
  \includegraphics{ui-prototypes/exports/Date management (Book Button pressed).png}
  \caption{Termin durchführen}
  \label{fig:date-management-book-button-pressed}
\end{figure}

\subsection{Administratoransicht}

\paragraph{Gruppen und Termine anlegen}
Mithilfe des Formulars kann der Administrator die initialen Gruppen und
Termine anlegen.
\begin{figure}
  \centering
  \includegraphics{ui-prototypes/exports/Setup.png}
  \caption{Gruppen und Termine anlegen}
  \label{fig:setup}
\end{figure}

\paragraph{Terminansicht}
In der Terminansicht kann sich der Administrator eine Übersicht über die
verfügbaren Termine veschaffen, Termine editieren und
Buchungen/Reservierungen verwalten.
\begin{figure}
  \centering
  \includegraphics{ui-prototypes/exports/Date management Admin.png}
  \caption{Terminansicht (Administrator)}
  \label{fig:date-management-admin}
\end{figure}

\subparagraph{Termin bearbeiten}
Durch die Schaltfläche ``Bearbeiten'', kann der Administrator den
Zeitraum und den Kommentar des Termins bearbeiten und den Termin
deaktivieren.
\begin{figure}
  \centering
  \includegraphics{ui-prototypes/exports/Date management Admin (Edit Button pressed, no Booking).png}
  \caption{Termin bearbeiten}
  \label{fig:edit-button-pressed-no-booking}
\end{figure}
Falls eine Buchung für den Termin besteht, kann der Administrator die
Buchung stornieren oder den Zeitraum sowie Prüfungsraum der Buchung
festlegen.
\begin{figure}
  \centering
  \includegraphics{ui-prototypes/exports/Date management Admin (Edit Button pressed, Booking present).png}
  \caption{Termin bearbeiten - Buchung}
  \label{fig:edit-button-pressed-booking}
\end{figure}
Falls eine Reservierung für den Termin besteht, kann der Administrator
die Reservierung stornieren.
\begin{figure}
  \centering
  \includegraphics{ui-prototypes/exports/Date management Admin (Edit Button pressed, Reservation present).png}
  \caption{Termin bearbeiten – Reservierung}
  \label{fig:edit-button-pressed-reservation}
\end{figure}
Falls der Termin deaktiviert ist, kann der Administrator den Termin
wieder aktivieren.
\begin{figure}
  \centering
  \includegraphics{ui-prototypes/exports/Date management Admin (Edit Button pressed, Deactivated).png}
  \caption{Termin bearbeiten – Deaktiviert}
  \label{fig:edit-button-pressed-deactivated}
\end{figure}

\paragraph{Gruppenansicht}
In der Gruppenansicht kann sich der Administrator einen Überblick über
den Zustand der Gruppen verschaffen, sowie Gruppen mithilfe der
entsprechenden Schaltflächen anlegen und löschen.
\begin{figure}
  \centering
  \includegraphics{ui-prototypes/exports/Group management.png}
  \caption{Gruppenansicht}
  \label{fig:group-management}
\end{figure}
